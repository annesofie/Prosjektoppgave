\chapter{Suggested method for importing FKB into OpenStreetMap}

\section{Method - Microtasking?}

% Går tilbake til metodene, hva kan vi lære i Norge hvor vi ikke har Mapbox. Norske FKB er bedre enn de man har importert i andre land
% Er dette beste metode?
%Hvordan dele opp i tasking områder? telle antall hus. 50-30-20 hus innenfor hver task? Dette er en utfordring, men er utenfor scopet til oppgaven. Komme med noen tanker rundt hvordan det kan løses. 


\section{FKB to OSM tagging}
%Hvilke tagger skal brukes for hva
%Skal vi bruke relations med 3D støtte? 

An example of a area representation of a building feature type is shown in listing \ref{eq:buildingfootpr}. This can be used as the building footprint when converting FKB to OSM. This will create the 2D modeling. As mentioned in section \ref{buildOSM}, is will be given a building=house tag. Building type (BYGGTYP NBR) is equal to 111. The value 111 represents a house. Using figure xx of the most common building types... Not all types used in FKB can directly translate to OSM. The tag-info page support searching for building values used in OSM. The most common FKB building types in OSM supported building values are shown in figure xx.

\lstset{
    language=XML,
    morekeywords={encoding,node, tag},
    label=eq:buildingfootpr,
    caption=Example of a area representation of a building feature type in SOSI. 
}
\begin{lstlisting}
.FLATE 715235:
..OBJTYPE Bygning
..KOMM 1601
..BYGGNR 182720836
..BYGGTYP_NBR 111
..BYGGSTAT TB
..KOPIDATA
...OMRÅDEID 1601
...ORIGINALDATAVERT "Trondheim kommune"
...KOPIDATO 20160502
..REF :166806
..NØ
703610900 55898600
\end{lstlisting}

\lstset{
    language=XML,
    morekeywords={encoding,node, tag},
    label=eq:buildingfootprline,
    caption=Example of a area representation of a building feature type in SOSI. 
}
\begin{lstlisting}
.KURVE 166806:
..OBJTYPE Takkant
..DATAFANGSTDATO 20100610
..VERIFISERINGSDATO 20150627
..REGISTRERINGSVERSJON "FKB" "4.01"
..KVALITET 24 19 0 24 23
..TRE_D_NIVÅ 2
..KOPIDATA
...OMRÅDEID 1601
...ORIGINALDATAVERT "Trondheim kommune"
...KOPIDATO 20160502
..NØH
703611202 55898706 1828 ...KP 1  
..NØH
703611328 55898445 1671
703610319 55897959 1671
703610193 55898220 1828 ...KP 1
..NØH
703610060 55898497 1671
703610540 55898728 1671 ...KP 1
..NØH
703610407 55899005 1671
703610668 55899130 1828 ...KP 1
..NØH
703610935 55899259 1671
703611065 55898990 1661 ...KP 1
..NØH
703611202 55898706 1828 ...KP 1
\end{lstlisting}


\section{Conversion using existing script}
% fkbbuilding.lua
%Lage utkast til en luafil, vise hvordan xml oppsettet blir 

The SOSI2OSM script was developed in 2013 and do not have any documentation. It is also difficult to install. It do not support SOSI files encoded in UTF-8, therefore the first step was to encode the FKB SOSI file into ISO8859-10 which is supported. %Er alle SOSI filer i UTF 8 idag?



\section{How to map 3D buildings}
%Her kan jeg lage et eksempel på hvordan 3D bygg kan moduleres i OSM XML format. 
In order to create a XML representation capable of modeling 3D buildings, a standard approach needs to be implemented. Members of the OpenStreetMap community, with interest in 3D mapping started in March 2012 to unite all the separated appraches to model 3D buildings with OSM \cite{OpenStreetMapm}. They arranged workshops, which resulted in a suggestion for a simple 3D building schema. Most 3D developers agreed on supporting a unified subset of tags in their programs \cite{OpenStreetMapl}. The participants on the workshop agreed on building outlines should be used for the most general area of a complex building and building parts used to describe the special parts with different heights or other attributes, etc. As mentioned in section \ref{buildOSM}, the building outline should be tagged with building=* and building parts with building:part=*. Building outline is either a closed way or a multi-polygon and represents the area of land covered by the union of all building parts, the building footprint \cite{OSMwikipage2016}. Attributes like address, name, height, etc. must be tagged on the building outline. The building outline is important because it provides the compatibility for 2D rendering software. 2D renderers ignore building:part=* tags and only displaying the building outline. A relation tagged with type=building should be used if there are more than  one building part. This groups the building outline and all building parts together, as mentioned in section \ref{buildOSM}. 

There are different tools visualizing 3D buildings. A list of which of them who supports the simple 3D building schema is located at the OSM simple 3D buildings wiki page, most tools accept this schema. 7 tools supports building:part=yes %http://wiki.openstreetmap.org/wiki/3D_Development/Tagging#Usage_Community
2D renderers ignore building:part=* tags, only displaying the building outline. 

An example of 3D building generated from FKB data is shown in listing xx. This use the simple 3D building schema. 

\section{Evaluation}