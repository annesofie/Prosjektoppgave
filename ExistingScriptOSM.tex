\chapter{OSM import methods}
1. Helautomatisk, scriptet import \\
2. Helmanuell import basert p� tracing av f.eks WMS-data \\
3. Guidet automatisk import (slik import av n50-data blir gjort i OSM i dag) \\
4. Import basert p� metodikken LA-buildings-prosjektet har gjort, og som beskrives av McAndrews (microtasking)

Conflation \href{http://wiki.openstreetmap.org/wiki/Conflation}{Wiki Conflation}

\subsection{Fully automatic import script}
A script that automatically imports big datasets into OpenStreetMap is not popular in the OSM community. Bulk imports is the process of uploading external data and were meant for initial/preliminary object class uploads, so only if the object were none existing in the area. Today the situation are changed, there are huge amounts of object types in OSM. Often bulk import overrides existing data which is one of the "dont do" points in the import guidelines list on the osm wiki page. OpenStreetMap do not have layers, so data on top of data makes it very difficult to organize and find the data. 

An example of a bulk import was the TIGER import. The Topologically Integrated Geographical Encoding and Reference system (TIGER) data was produced by the US Census Bureau and is a public domain data source. The bulk import was completed in 2007, populating the nearly empty map of the United States. The TIGER data was not perfect and had it's faults, but it was better than no data at all \cite{Willis2008}. 

On the TIGER OpenStreetMap wikipage, last updated August 2016, they say it is unlikely that the TIGER data ever will be imported again. The main reason is the growing US mapping community, their mapping is often better than the TIGER data. "Do not worry about getting your work overwritten by new TIGER data. Go map!" \cite{WikiOSMTIGER2007}. A new bulk import with updated TIGER data can overwrite existing, more precise data. The TIGER data are of variable quality, poor road alignment is a huge problem and also wrong highway classification. Many hours of volunteer work could be lost and this is something the community want to avoid. The bulk import in 2007 got the United States on the OSM-Map and saved the mapping community a lot of time finding road names etc. "TIGER is a skeleton on which we can build some much better maps" \cite{Willis2007}. On the negative side the project kept the US mappers away, they were told for years that their work was no longer needed after the TIGER upload was complete. But the presence of TIGER data ended up not eliminating the need for volunteer help, they needed help fixing errors like the poor road alignments. 

Bulk imports are overall not recommended today, but have been helpful as well. In the Netherlands bulk imports have meet little resistance, mainly because the imports are done by dedicated OpenStreetMap mappers who knew the OSM import guidelines.  Arguments in favor of bulk imports say that a map that already contains some information is easier to work on and can help lower the entry barriers for new contributors. Another argument is that a almost complete map is more attractive for potential users, that again can encourage more use of OSM data in professional terms \cite{Exelvan2010}. But a huge minus to bulk imports are the data aging, since the data being imported often already is a few years old and updating it takes time, often years. Between the time of first import and update the community have fixed bugs, added important metadata, the community would not want to loose that data/information.   
%has common errors like wrong road classifications, adds roads that no longer or never existed, railway and roads 
%moved so they intersect when they do not. 

\subsection{Fully manually import}
Very time consuming, and also the mapping quality depends on the image resolution. 


\subsection{Guided automatic import}
Fully automatic import of huge amounts of data is discouraged, so another approach is guided automatic import. The OSM community encourages people to import only small amounts of data at a time and only after validation and correcting errors. This method was used when the OSM-community in Norway got approval from the Norwegian map authority to import N50 data which is the official topographic map of Norway. Their import process is described on the OSM wikipage. It says that they will import one municipality at a time. Each municipality dataset will be divided up in a .15 deg times .15 deg grid changeset and imported grid by grid*. The N50 import was an community import, but only experienced user were encouraged to import the data. 

Sosi to osm


\subsection{Import based on LA-buildings methodology}
